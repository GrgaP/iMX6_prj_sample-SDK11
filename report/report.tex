\documentclass[conference,a4paper]{IEEEtran}

\usepackage[utf8]{inputenc}
\usepackage[english]{babel}
\usepackage[T1]{fontenc}
\usepackage{graphicx,xcolor}
\usepackage[hidelinks]{hyperref}
\usepackage{amsfonts}

\hyphenation{net-works}

\begin{document}
\title{Porting RIOT-OS to the RIoTboard}

\author{\IEEEauthorblockN{Jakob Lennart Dührsen and Leon Martin George}
\IEEEauthorblockA{Institute of Computer Science\\
Freie Universität Berlin\\
Takustrasse 9, 14195 Berlin, Germany\\
Email: \{lennart.duehrsen, leon.george\}@fu-berlin.de}}

\maketitle

\begin{abstract}
This is a report on the process of adding basic support for the RIoTboard to 
RIOT-OS as part of the »Softwareproject Telematics 2014« at Freie Universität 
Berlin.
The port is based on the SDK for the i.MX6 platform (the heart of the 
RIoTboard), and at the end of the project was in a state where LEDS, UART,
timers and interrupts have interfaces to RIOT-OS.
\end{abstract}

\IEEEpeerreviewmaketitle

\section{Introduction}
In~\cite{Hofstadter:1979} Douglas Hofstaedter applies Godel's seminal contribution to modern mathematics to the study of the human mind and the development of artificial intelligence.


\section{Preparing For The Port}

In order to complete the project successfully, we agreed on these goals with the
teaching-staff:

\begin{itemize}
\item The RIoT-OS-wiki has pages on working with the hardware
\item RIoT-OS compiles for the RIoTboard
\item Interfaces to timers, UART and interrupts is implemented
\end{itemize}

As we had no idea how hard it was going to be, we were very generous with the curfews
for the different steps.

\subsection{The Hardware}
The RIoTboard is a rather unusual board for the RIoT-OS to run on.
It is based on the "i.MX6Solo" freescale-architecture, has an ARM cortex-A9, many
different interfaces and supposed to be used mainly by developers.


\subsection{Software Running on the RIoTboard}
On the boards website the manufacturer, embest-tech, offers binary-images and instructions
for running android or ubuntu, of which android is installed by default. Both operating
systems rely on u-boot to do low-level hardware-initialisation. The source-code of those
ports is available in moderately-hard-to-find repositories on the internet and
the code itself could somehow be re-useable.

Due to their structure, the linux-code and the u-boot-fork were useful in different ways:
u-boot requires the seperation of syscalls, the "flash-header" - which helped understand
the way the i.MX6 boots and linker script.
So with u-boot embest-tech provides three files for those and one giant source-file for
all the rest.
The linux-port seems better structured but due to us lacking the knowledge about
the kernel it was harder to use any parts of the source..
Also, as far as initialisation goes, linux seems to repeat some of the steps that have
already been done by u-boot and does some differently.

The focus on developers is somehow restricted to those two operating systems. There is no
official support apart from discussions on a channel of an IoT/embedded-focused developer
forum.

\section{Running our own Software}

The simplest thing we could get to run on the RIoTboard was u-boot: Either via the serial
connection, where the freshly transferred u-boot and an initramfs could be used to
receive files or binary that could then be stored on the board, or by writing the binary
onto a SD-card. From within the running u-boot it was possible to execute a method from
a cross-compiled object-file that simply returned an integer value.

We were trying to figure out how to use u-boot for booting and supplying the standard
library but all the files were spread all over the u-boot-source-directory.
As we were realising how difficult it was going to be to not only have a functional
binary based on our own code but also integrate the existing structure into the
make-system of RIoT-OS, we didn't think we would be able to meet any goal beyond
supplying a rudimentary framework so that coding can start in the RIoT-OS-codebase.

Luckily, freescale supplies a SDK for the i.MX6-platform that is able to run on their
reference-boards. We agreed with the teaching-staff to rather use the SDK because it is
closer to just developing applications in a C-manner and the source-tree is
well-structured without making it too hard to find a particular piece of code and also,
there are hardly any source-files with irrelevant code.

Then again, sadly, none of the targets for the SDK seemed to produce code that ran on the
RIoTboard.

\section{Booting the i.MX6}

While trying to find a simple way of flashing the ROM of the cortex-A9 we learned that
freescale does not want anybody to flash the CPU directly. Instead there is a program on
the ROM that checks hardware-signals called \textbf{FUSE}s to decide how the CPU boots.
Essentially, they allow to boot from one of four \textit{USDHC}-block-devices or a
serial-download-mode. In theory, \textbf{FUSE}s can be used to change other
parameters for booting.

In case of the RIoTboard the first three \textit{USDHC}-lines are 4 GB of internal eMMC,
the SD- and the $\mu$SD-card-slot. The serial interface is wired to the mini-USB-port
next to the ethernet-port and UART2-pins.
Slightly more-detailed information can be found in the gitlab-repository mentioned in
the introduction (\textit{$briefings/boot-switches\_and\_-modes$})

\section{Running our own Software}

The simplest thing we could get to run on the RIoTboard was u-boot: Either via the serial
connection, where the freshly transferred u-boot and an initramfs could be used to
receive files or binary that could then be stored on the board, or by writing the binary
onto a SD-card. From within the running u-boot it was possible to execute a method from
a cross-compiled object-file that simply returned an integer value.
There also is a brief step-through for building u-boot and in the repository for this
port (\textit{$briefings/build\_and\_run\_with\_uboot\_from\_sd-card$}).

We were trying to figure out how to use u-boot for booting and supplying the standard
library but all the files were spread all over the u-boot-source-directory.
As we were realising how difficult it was going to be to not only have a functional
binary based on our own code but also integrate the existing structure into the
make-system of RIoT-OS, we didn't think we would be able to meet any goal beyond
supplying a rudimentary framework so that coding can start in the RIoT-OS-codebase.

By then we were behind schedule as we could not get anything to run that was not built
on top of and run from within u-boot.


\section{Integrating the i.MX6-Platform-SDK into RIoT-OS}

Luckily, freescale supplies a SDK for the i.MX6-platform that is able to run on their
reference-boards. We agreed with the teaching-staff to rather use the latest version of
this SDK (1.1) because it is closer to developing applications in a C-manner, a bit more
minimalistic and the source-tree is well-structured without making it too hard to find a
particular piece of code and also, there are hardly any source-files that are irrelevant
to this particular port.
\\
Then again, sadly, none of the targets for the SDK seem to produce code that runs on the
RIoTboard and was not clear whether we would be able to produce anything using the SDK
but it seemed like the right way to go because of its rather clean interfaces.

The i.MX6-platform-SDK brings a \textit{make}-structure that has applications as targets
and allows a target-board to be specified via parameters.
After the board has been chosen, CPU- and board-specific code will be included based on
defines.
These defines were passed as parameters to \textbf{make}. We did not see any sense in
adopting this pattern in RIoT-OS as it would have made changes to the
\textit{make}-system necessary.
In order to avoid this problem, we picked the defines relevant to the i.MX6Solo and the
RIoTboard and changed the code manually as if the preprocessor had run.

This has major implications for future ports to i.MX6-variants other than the
Solo/DualLite: Every source- or header-file of the un-modified SDK containing
\textit{\#ifdef}s or \textit{\#ifndef} potentially contains code important for a successful
port.
Because of our concerns about completing the project in time we decided not to develop a
proper concept to deal with this issue.
We do however have a proposal which might turn out to work quiet nice without putting to
much effort into it:
\begin{itemize}
\item Add prefixes to SDK-macro names (e.g. \textbf{IMX6\_SDK\_})
\item Define the necessary macros in the \textit{board-sdk.h}
\item Include the header file in every SDK-file.
\end{itemize}
We could implement this relatively simple technique but we dont have the means to test
it and expect a better way to exist.

The next step was to correctly build the SDK when \textbf{make} was run from within RIoT.
As even the \textit{Makefile}s of the SDK worked with defines, we decided to scrap and
re-write them. Understanding the \textit{make}-structure of RIoT-OS and creating
\textit{Makefile}s for the SDK took us about a week.

Even though practically everything apart from the low-level initialisation-code needed
minor to moderate adjustments we were able to skip the step of building a minimal program
and go directly to builing the hello-world-program of RIoT-OS.

\section{Enabling Interfaces of the RIoTboard within RIoT-OS}

As all the components of the RIoTboard have a driver in the platform-SDK we did not
expect these steps to be too hard. However, we heavily underestimated the importance of
the IOMuxer on the board. This component has no driver in the SDK, as it is not supposed
to be manipulated while the program is running. Instead you are supposed to use a
windows-only tool called "IOMux-Tool" which allows muxing pins by clicking on the desired
functions.
It automatically checks for conflicts and can generate source-files containing macros
that will work on the right registers without you having to worry about muxer-config
after you have used the tool.
We added a way to correctly alter and copy the generated source files into the right
directories. See the corresponding briefing in our repository for instructions on using
it (\textit{$briefings/muxer\_config$}).
If you are not using this script, please make sure the IoMuxDesign.xml always matches the
set of macros currently in the riotboard-directory of RIoT-OS.

\subsection{LED}
During our preparation for running a minimalistic C-program on the RIoTboard we had
looked up how the connection of the CPU to the LEDs work.
With this knowledge is was very easy to write code for flashing the LEDs using the SDKs
GPIO-driver.
The program we were running then needed only the GPIO-pins to be configured in the muxer.

\subsection{UART}
We also spent a large amount of time on enabling UART2 for I/O-support:
Although the UART-initialisation ran on the right registers and nothing seemed to be
wrong with with the code itself, the UART did not do anything. Neither could we receive
any output nor did the board process any of the signals we sent.
Without the UART our only means for testing were the LEDs. When we thought it could be
only the output-channel that was not working, we tried toggling the LED upon reception.

We tried replacing the default SDK-code for initialisation with u-boot-code that was
altered to use the SDK-macros. This version could be fixed by rearranging the
instructions in a different order. We did not understand why this worked.

In the end, we had - by accident - enabled a mechanism that would forbid using the UART
from being used while its buffer is overloaded. This mechanism stopped the UART from
working altogether because the other side was not configured to use it.
For some reason this was also activated in the default SDK-code.

\subsection{Timers}
Having missed our last deadline we decided to get help understanding the timer-interface
of RIoT-OS before we started working on this subsystem. We were told there was a clean
implementation using another layer of timer-abstraction in the code for another
processor.
But as soon as we started implementing this new lower-level interface we noticed that the
SDK already offered a higher-level abstraction and scrapped the code we had so far.
There were only slight conceptual problems with mapping the SDK-driver to the
RIoT-OS-interface because of the SDK offering slightly differenct functions than RIoT-OS
expects.

\subsection{Interrupts}

The SDK already offers a way to map interrupt-callback-routines to interrupt-IDs.
We do not think this mechanism should be used directly. The drivers probably are the
right way to use interrupts.
This practically gave us a milestone for free. Apart from that, we would not have had the
time to do anything else before cleaning the code.
And the code being useable in the future is more useful than another interface.


\section{pull requests}

At the time of writing we have one failed, a successful and another pending pull request:
\begin{itemize}
\item \#1355 was closed in favour of \#1359. Leon had trouble keeping the git-log tidy.
This would have allowed using LD for linking by changing the global Makefile and the
Makefile of any board not using LD (so far: all except the RIoTboard).
\item \#1359 was merged - with the help of staff members and RIoT-maintainers.
This basically does the same as \#1355 but leaves other Makefiles untouched and
assuming GCC to be used by default. Boards using LD have to explicitely supply a
variable.
\item \#1411 is still pending and we do not know whether it will be merged.
It's purpose is to bring the software-project to a conclusion and add support for
the riotboard and the i.MX6-SDK.
\end{itemize}

Despite the struggles during the work so far we remain motivated to continue on this
port.
THIS GOES TO ANOTHER TEX FILE:
Should we ever do another project we will focus more on maintaining a clean codebase than
on adding as much functionality at once.

\section{TODO}

This section of the report is about steps that should be taken not only to finish the sofware-project but also have a port that can be merged.

These are the steps which should be considered:
\begin{itemize}
\item move files from the packages back into boards/riotboard and cpu/imx6
\item remove headers and iomux-config for components that will not be used
\item strip the remaining header files of macros, bitmasks and -offsets
\item fix license-headers of remaining untouched source-files or adjust check.sh to tolerate the BSD-style licenses
\end{itemize}

Because there are about twenty register-files for each variant of the i.MX6 and they contain thousands of lines of code we decided not to do those as for the software-project itself.


\bibliographystyle{unsrt}
\bibliography{references}

\end{document}
