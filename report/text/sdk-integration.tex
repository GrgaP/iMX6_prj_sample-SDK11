\section{Integrating the i.MX6-Platform-SDK into RIoT-OS}

Luckily, freescale supplies a SDK for the i.MX6-platform that is able to run on their
reference-boards. We agreed with the teaching-staff to rather use the latest version of
this SDK (1.1) because it is closer to developing applications in a C-manner, a bit more
minimalistic and the source-tree is well-structured without making it too hard to find a
particular piece of code and also, there are hardly any source-files that are irrelevant
to this particular port.
\\
Then again, sadly, none of the targets for the SDK seem to produce code that runs on the
RIoTboard and was not clear whether we would be able to produce anything using the SDK
but it seemed like the right way to go because of its rather clean interfaces.

The i.MX6-platform-SDK brings a \textit{make}-structure that has applications as targets
and allows a target-board to be specified via parameters.
After the board has been chosen, CPU- and board-specific code will be included based on
defines.
These defines were passed as parameters to \textbf{make}. We did not see any sense in
adopting this pattern in RIoT-OS as it would have made changes to the
\textit{make}-system necessary.
In order to avoid this problem, we picked the defines relevant to the i.MX6Solo and the
RIoTboard and changed the code manually as if the preprocessor had run.

This has major implications for future ports to i.MX6-variants other than the
Solo/DualLite: Every source- or header-file of the un-modified SDK containing
\textit{\#ifdef}s or \textit{\#ifndef} potentially contains code important for a successful
port.
Because of our concerns about completing the project in time we decided not to develop a
proper concept to deal with this issue.
We do however have a proposal which might turn out to work quiet nice without putting to
much effort into it:
\begin{itemize}
\item Add prefixes to SDK-macro names (e.g. \textbf{IMX6\_SDK\_})
\item Define the necessary macros in the \textit{board-sdk.h}
\item Include the header file in every SDK-file.
\end{itemize}
We could implement this relatively simple technique but we dont have the means to test
it and expect a better way to exist.

The next step was to correctly build the SDK when \textbf{make} was run from within RIoT.
As even the \textit{Makefile}s of the SDK worked with defines, we decided to scrap and
re-write them. Understanding the \textit{make}-structure of RIoT-OS and creating
\textit{Makefile}s for the SDK took us about a week.

Even though practically everything apart from the low-level initialisation-code needed
minor to moderate adjustments we were able to skip the step of building a minimal program
and go directly to builing the hello-world-program of RIoT-OS.
