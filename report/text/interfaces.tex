\section{Enabling Interfaces of the RIoTboard within RIoT-OS}

As all the components of the RIoTboard have a driver in the platform-SDK we did not
expect these steps to be too hard. However, we heavily underestimated the importance of
the IOMuxer on the board. This component has no driver in the SDK, as it is not supposed
to be manipulated while the program is running. Instead you are supposed to use a
windows-only tool called "IOMux-Tool" which allows muxing pins by clicking on the desired
functions.
It automatically checks for conflicts and can generate source-files containing macros
that will work on the right registers without you having to worry about muxer-config
after you have used the tool.
We added a way to correctly alter and copy the generated source files into the right
directories. See the corresponding briefing in our repository for instructions on using
it (\textit{$briefings/muxer\_config$}).
If you are not using this script, please make sure the IoMuxDesign.xml always matches the
set of macros currently in the riotboard-directory of RIoT-OS.

\subsection{LED}
During our preparation for running a minimalistic C-program on the RIoTboard we had
looked up how the connection of the CPU to the LEDs work.
With this knowledge is was very easy to write code for flashing the LEDs using the SDKs
GPIO-driver.
The program we were running then needed only the GPIO-pins to be configured in the muxer.

\subsection{UART}
We also spent a large amount of time on enabling UART2 for I/O-support:
Although the UART-initialisation ran on the right registers and nothing seemed to be
wrong with with the code itself, the UART did not do anything. Neither could we receive
any output nor did the board process any of the signals we sent.
Without the UART our only means for testing were the LEDs. When we thought it could be
only the output-channel that was not working, we tried toggling the LED upon reception.

We tried replacing the default SDK-code for initialisation with u-boot-code that was
altered to use the SDK-macros. This version could be fixed by rearranging the
instructions in a different order. We did not understand why this worked.

In the end, we had - by accident - enabled a mechanism that would forbid using the UART
from being used while its buffer is overloaded. This mechanism stopped the UART from
working altogether because the other side was not configured to use it.
For some reason this was also activated in the default SDK-code.

\subsection{Timers}
Having missed our last deadline we decided to get help understanding the timer-interface
of RIoT-OS before we started working on this subsystem. We were told there was a clean
implementation using another layer of timer-abstraction in the code for another
processor.
But as soon as we started implementing this new lower-level interface we noticed that the
SDK already offered a higher-level abstraction and scrapped the code we had so far.
There were only slight conceptual problems with mapping the SDK-driver to the
RIoT-OS-interface because of the SDK offering slightly differenct functions than RIoT-OS
expects.

\subsection{Interrupts}

The SDK already offers a way to map interrupt-callback-routines to interrupt-IDs.
We do not think this mechanism should be used directly. The drivers probably are the
right way to use interrupts.
This practically gave us a milestone for free. Apart from that, we would not have had the
time to do anything else before cleaning the code.
And the code being useable in the future is more useful than another interface.
