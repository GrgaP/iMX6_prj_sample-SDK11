\section{Enabling Interfaces of the RIoTboard within RIoT-OS}

As all the components of the RIoTboard have a driver in the platform-SDK we did not
expect these steps to be too hard. However, we heavily underestimated the importance of
the IOMuxer on the board. This component has no driver in the SDK, as it is not supposed
to be manipulated while the program is running. Instead you are supposed to use a
windows-only tool called "IOMux-Tool" which allows muxing pins by clicking on the desired
functions.
It automatically checks for conflicts and can generate source-files containing macros
that will work on the right registers without you having to worry about muxer-config
after you have used the tool.
We added a way to correctly alter and copy the generated source files into the right
directories. See the corresponding briefing in our repository for instructions on using
it (\textit{$briefings/muxer\_config$}).
If you are not using this script, please make sure the IoMuxDesign.xml corresponds
to the set of macros currently in the riotboard-directory of RIoT-OS.

\subsection{UART}

In the end, we had - by accident - enabled a mechanism that would forbid using the UART
from being used while its buffer is overloaded.

\subsection{Timers}

\subsection{Interrupts}

The SDK already offers a way to map interrupt-callback-routines to interrupt-IDs.
We do not think this mechanism should be used directly. The drivers probably are the
right way to use interrupts.
This practically gave us a milestone for free. Apart from that, we would not have had the
time to do anything else before cleaning the code.
