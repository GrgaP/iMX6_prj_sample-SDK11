\section{Booting the i.MX6}

While trying to find a simple way of flashing the ROM of the cortex-A9 we learned that
freescale does not want anybody to flash the CPU directly. Instead there is a program on
the ROM that checks hardware-signals called \textbf{FUSE}s to decide how the CPU boots.
Essentially, they allow to boot from one of four \textit{USDHC}-block-devices or a
serial-download-mode. In theory, \textbf{FUSE}s can be used to change other
parameters for booting.

In case of the RIoTboard the first three \textit{USDHC}-lines are 4 GB of internal eMMC,
the SD- and the $\mu$SD-card-slot. The serial interface is wired to the mini-USB-port
next to the ethernet-port and UART2-pins.
Slightly more-detailed information can be found in the gitlab-repository mentioned in
the introduction (\textit{$briefings/boot-switches\_and\_-modes$})

At startup, the program on the ROM loads a chunk from the beginning of the chosen boot
device and checks the existence of an IVT (Image-Vector-Table). This table contains the
start address of the program-code, a checksum and a pointer to DCD
(Device-Configuration-Data). DCD is used to initialise register-memory and consists of
pairs of addresses and 32-bit values which will be written to the registers at the
preceding address. Details on this can be found in the i.MX6 processor reference manual
in the chapter "Image Vector Table and Boot Data".
The offset of the IVT for the RIoTboard - as all devices are connected via UDSHC - is 1KB
 - which allows a MBR to be placed on the device also - and the size of the initially
loaded chunk is 4KB, so the first partition should not start before \$1400. Too quickly
get an image to run on the RIoTboard follow the steps in
\textit{$build\_and\_run\_with\_uboot\_from\_sd-card$}.

\section{Running our own Software}

The simplest thing we could get to run on the RIoTboard was u-boot: Either via the serial
connection, where the freshly transferred u-boot and an initramfs could be used to
receive files or binary that could then be stored on the board, or by writing the binary
onto a SD-card. From within the running u-boot it was possible to execute a method from
a cross-compiled object-file that simply returned an integer value.
The brief step-through for building u-boot and running it on the RIoTboard in our
repository also contains information on the object-file-test
(\textit{$briefings/build\_and\_run\_with\_uboot\_from\_sd-card$}).

We were trying to figure out how to use u-boot for booting and supplying the standard
library but all the files were spread all over the u-boot-source-directory.
As we were realising how difficult it was going to be to not only have a functional
binary based on our own code but also integrate the existing structure into the
make-system of RIoT-OS, we didn't think we would be able to meet any goal beyond
supplying a rudimentary framework so that coding can start in the RIoT-OS-codebase.

By then we were behind schedule as we could not get anything to work that was not built
on top of and run from within u-boot.
